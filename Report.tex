% VDE Template for EUSAR Papers
% Provided by Barbara Lang und Siegmar Lampe
% University of Bremen, January 2002
% English version by Jens Fischer
% German Aerospace Center (DLR), December 2005
% Additional modifications by Matthias Wei{\ss}
% FGAN, January 2009

%-----------------------------------------------------------------------------
% Type of publication
\documentclass[a4paper,10pt]{article}
%-----------------------------------------------------------------------------
% Other packets: Most packets may be downloaded from www.dante.de and
% "tcilatex.tex" can be found at (December 2005):
% http://www.mackichan.com/techtalk/v30/UsingFloat.htm
% Not all packets are necessarily needed:
\usepackage[T1]{fontenc}
\usepackage[latin1]{inputenc}
%\usepackage{ngerman} % in german language if required
\usepackage[nooneline,bf]{caption} % Figure descriptions from left margin
\usepackage{times}
\usepackage{multicol}
\usepackage{amsmath}
\usepackage{amssymb}
\usepackage[dvips]{graphicx}
\usepackage{epsfig}
\input{tcilatex}
%-----------------------------------------------------------------------------
% Page Setup
\textheight24cm \textwidth17cm \columnsep6mm
\oddsidemargin-5mm                 % depending on print drivers!
\evensidemargin-5mm                % required margin size: 2cm
\headheight0cm \headsep0cm \topmargin0cm \parindent0cm
\pagestyle{empty}                  % delete footer and header
%----------------------------------------------------------------------------
% Environment definitions
\newenvironment*{mytitle}{\begin{LARGE}\bf}{\end{LARGE}\\[1.5ex]}%
\newenvironment*{myabstract}{\begin{Large}\bf}{\end{Large}\\[2.5ex]}%
%-----------------------------------------------------------------------------
% Using Pictures and tables:
% - Instead "table" write "tablehere" without parameters
% - Instead "figure" write "figurehere " without parameters
% - Please insert a blank line before and after \begin{figuerhere} ... \end{figurehere}
%
% CAUTION:   The first reference to a figure/table in the text should be formatted fat.
%
\makeatletter
\newenvironment{tablehere}{\def\@captype{table}}{}
\newenvironment{figurehere}{\def\@captype{figure}\vspace{2ex}}{\vspace{2ex}}
\makeatother



%%%%%%%%%%%%%%%%%%%%%%%%%%%%%%%%%%%%%%%%%%%%%%%%%%%%%%%%%%%%%%%%%%%%%%%%%%%%%%
\begin{document}

% Please use capital letters in the beginning of important words as for example
\begin{mytitle} A LaTeX Report Template. \end{mytitle}
%
% Please do not insert a line here
%
\\
Marini Matteo\\
Matr. 724626\\
\hspace{10ex}

\begin{myabstract} Abstract \end{myabstract}

Multicore architectures, which contain multiple processing cores on a single chip, have been adopted by most chip manufacturers.
Dual-core chips are commonplace, and numerous four- and eight-core options exist. In the coming years, per-chip core counts will continue to increase.
The major obstacle to use multicores for real-time applications is that we may not predict and provide any guarantee on real-time properties of embedded software on such platforms; the way of handling the on-chip shared resources such as L2 cache may have a significant impact on the timing
predictability. For this reason, it is necessary introduce new scheduling algorithms that consider how scheduled tasks use shared cache, in briefly it is necessary  to develop "cache-aware" scheduling algorithms. In this report, we show a model that represents cache-aware scheduling problem and some examples of cache-aware scheduling algorithms.

\vspace{4ex}	% Please do not remove or reduce this space here.
\begin{multicols}{2}

%%%%%%%%%%%%%%%%%%%%%%%%%%%%%%%%%%%%%%%%%%%%%%%%%%%%%%%%%%%%%%%%%%%%%%%%%%%%%
\section{Introduction}

TODO qui non faccio cenni a come sono fatte le architetture, lo metto nel report delle cache

Multicore architectures, which contain multiple processing cores on a single chip, have been adopted by most chip manufacturers.
Dual-core chips are commonplace, and numerous four and eight core options exist. In the coming years, per-chip core counts will continue to
increase for example, Sun plans to ship its 16-core "Rock" processor by the end of 2009, and Intel has claimed that it will release 80-core
chips as early as 2013. The shift to multicore technologies is a watershed event, as it fundamentally changes the "standard" computing
platform in many settings to be a multiprocessor. In most multicore platforms, different cores share onchip caches. Without effective
management by the scheduler, such caches can cause thrashing that severely degrades system performance. In fact, the issue of efficient cache
usage on multicore platforms is one of the most important problems with which chip makers are currently grappling.
It is predicted that multicores will be increasingly used in future embedded systems for high performance and low energy consumption. The
major obstacle is that we may not predict and provide any guarantee on real-time properties of embedded software on such platforms due to the on-chip
shared resources. Shared caches such as L2 or L3 cache are among the most critical resources on multicores, which severely degrade the timing
predictability of multicore systems due to the cache contention between cores. For single processor systems, there are well-developed techniques 
for timing analysis of embedded software. Using these techniques, the worst-case execution time (WCET) of real-time tasks may be estimated, and
then used for system-level timing analyses like schedulability analysis. One major problem in WCET analysis is how to predict the cache behavior,
since different cache behaviors (cache hit or miss) will result in different execution times of each instruction.
Unfortunately the existing techniques for single processor platforms are not applicable for multicores with shared caches. The reason is
that a task running on one core may evict the useful L2 cache content belonging to a task running on another core and therefore the
Worst-Case Execution Time (WCET) of one task can not be estimated in isolation from the other tasks as for single processor systems. 
Essentially, the challenge is to model and predict the cache behavior for concurrent programs (not sequential programs as for the case of 
single processor systems) running on different cores.

%%%%%%%%%%%%%%%%%%%%%%%%%%%%%%%%%%%%%%%%%%%%%%%%%%%%%%%%%%%%%%%%%%%%%%%%%%%%%
\section{The first section: scheduling model}





% We suggest the use of JabRef for editing your bibliography file (Report.bib)
\bibliographystyle{splncs}
\bibliography{Report}

\end{multicols}
\end{document}
