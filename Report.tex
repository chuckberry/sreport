% VDE Template for EUSAR Papers
% Provided by Barbara Lang und Siegmar Lampe
% University of Bremen, January 2002
% English version by Jens Fischer
% German Aerospace Center (DLR), December 2005
% Additional modifications by Matthias Wei{\ss}
% FGAN, January 2009

%-----------------------------------------------------------------------------
% Type of publication
\documentclass[a4paper,10pt]{article}
%-----------------------------------------------------------------------------
% Other packets: Most packets may be downloaded from www.dante.de and
% "tcilatex.tex" can be found at (December 2005):
% http://www.mackichan.com/techtalk/v30/UsingFloat.htm
% Not all packets are necessarily needed:
\usepackage[T1]{fontenc}
\usepackage[latin1]{inputenc}
%\usepackage{ngerman} % in german language if required
\usepackage[nooneline,bf]{caption} % Figure descriptions from left margin
\usepackage{times}
\usepackage{multicol}
\usepackage{amsmath}
\usepackage{amssymb}
\usepackage[dvips]{graphicx}
\usepackage{epsfig}
\input{tcilatex}
%-----------------------------------------------------------------------------
% Page Setup
\textheight24cm \textwidth17cm \columnsep6mm
\oddsidemargin-5mm                 % depending on print drivers!
\evensidemargin-5mm                % required margin size: 2cm
\headheight0cm \headsep0cm \topmargin0cm \parindent0cm
\pagestyle{empty}                  % delete footer and header
%----------------------------------------------------------------------------
% Environment definitions
\newenvironment*{mytitle}{\begin{LARGE}\bf}{\end{LARGE}\\[1.5ex]}%
\newenvironment*{myabstract}{\begin{Large}\bf}{\end{Large}\\[2.5ex]}%
%-----------------------------------------------------------------------------
% Using Pictures and tables:
% - Instead "table" write "tablehere" without parameters
% - Instead "figure" write "figurehere " without parameters
% - Please insert a blank line before and after \begin{figuerhere} ... \end{figurehere}
%
% CAUTION:   The first reference to a figure/table in the text should be formatted fat.
%
\makeatletter
\newenvironment{tablehere}{\def\@captype{table}}{}
\newenvironment{figurehere}{\def\@captype{figure}\vspace{2ex}}{\vspace{2ex}}
\makeatother



%%%%%%%%%%%%%%%%%%%%%%%%%%%%%%%%%%%%%%%%%%%%%%%%%%%%%%%%%%%%%%%%%%%%%%%%%%%%%%
\begin{document}

% Please use capital letters in the beginning of important words as for example
\begin{mytitle} A LaTeX Report Template. \end{mytitle}
%
% Please do not insert a line here
%
\\
Marini Matteo\\
Matr. 724626\\
\hspace{10ex}

\begin{myabstract} Abstract \end{myabstract}
Multicore architectures, which contain multiple processing cores on a single chip, have been adopted by most chip manufacturers.
Dual-core chips are commonplace, and numerous four- and eight-core options exist. In the coming years, per-chip core counts will continue to increase.
The major obstacle to use multicores for real-time applications is that we may not predict and provide any guarantee on real-time properties of embedded software on such platforms; the way of handling the on-chip shared resources such as L2 cache may have a significant impact on the timing
predictability. For this reason, it is necessary introduce new scheduling algorithms that consider how scheduled tasks use shared cache, in briefly it is necessary  to develop "cache-aware" scheduling algorithms. In this report, we show a model that represents cache-aware scheduling problem and some examples of cache-aware scheduling algorithms.

\vspace{4ex}	% Please do not remove or reduce this space here.
\begin{multicols}{2}

%%%%%%%%%%%%%%%%%%%%%%%%%%%%%%%%%%%%%%%%%%%%%%%%%%%%%%%%%%%%%%%%%%%%%%%%%%%%%
\section{Introduction}


%%%%%%%%%%%%%%%%%%%%%%%%%%%%%%%%%%%%%%%%%%%%%%%%%%%%%%%%%%%%%%%%%%%%%%%%%%%%%
\section{The Second Section}

Lorem ipsum dolor sit amet, consectetur adipiscing elit.  Aenean magna. Nunc non
ante eget nibh condimentum tempor. Nullam ullamcorper lectus eget mauris. Nam
neque orci; rhoncus at, pulvinar quis, elementum sit amet, turpis. Mauris
posuere nisi ut justo. Morbi non lorem vitae mauris interdum faucibus.
Vestibulum ut sapien in augue faucibus fringilla. Vestibulum ante ipsum primis
in faucibus orci luctus et ultrices posuere cubilia Curae; Etiam vestibulum
fringilla libero. Curabitur libero diam, hendrerit sit amet, ornare eget,
imperdiet vel, purus!


% We suggest the use of JabRef for editing your bibliography file (Report.bib)
\bibliographystyle{splncs}
\bibliography{Report}

\end{multicols}
\end{document}
